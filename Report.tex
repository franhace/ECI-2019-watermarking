\documentclass[11pt]{article}

%\usepackage{graphics}
\usepackage{graphicx,amsmath,amssymb}
%\usepackage[spanish]{babel} 
\usepackage[utf8]{inputenc}
\textwidth 16.5cm
\textheight 25.0cm
\voffset -2.9cm
\hoffset -2.0cm

\begin{document}
\pagestyle{empty}

\begin{center}

{\Large {\bf Report of ECI 2019 Course: `` Introduction to Steganography and Watermarking ``}} \\

\bigskip
{\large \bf Assignment E.316-N}
\end{center}

\begin{center}
Acha Francisco, Caldo Juan Pablo, Cardenas Rodrigo, Castagna Franco and Remedi Elias.
\end{center}

%% ABSTRACT!! (At the end)

\section{Introduction}

%Define steganography
Steganography is the procedure of insert information inside a data source without changing its perceptual quality.
Digital steganography uses digital data sources as a cover for hidden information. Examples of digital covers
are digital text files, image files and sound files among others.  

%Introduce context for steganographic methods on images
In particular for digital image based steganography the pixel intensity is usually used for encoding information [REF]
but other approaches are also widely used such as embedding information in the frequency domain.

%Review steganography over images (main features)
%Tell about software available for digital image-steganography
There are available many software tools 

%Describe goals of current assignment
In this report, five steganographic tools that hides text into a digital image were chosen to perform an assessment 
in terms of imperceptibility of the stego-image, capacity and robustness.

\section{Materials and methods}



%Dataset characterization
Since we want to evaluate performance of steganographic tools that hide text into an image, a image dataset is needed.
We built a dataset containg images 20 of four types: N-type (landscapes and open nature), S-type (still life), P-type (portraits) and
T-type (text). The complete dataset is then 80 images in total. N, S, P-type images were obtained and selected from Google images search engine queries.
Namely, keywords for queries were \textit{landscapes}, \textit{still life} and \textit{portrait} respectively.
Right usage for the images was selected such that results were labeled for noncommercial reuse, and size of the images was set in 
medium [NOTA AL PIE DE LA FECHA]. Text images were collected from research papers by exporting pages as jpeg images.
Table [REF] summarizes some basic features of the dataset used such as mean image size and mean file size. All the
images in the dataset were stored as jpeg format. DECIR AHORA EL TAMAÑO MEDIO, Y LA MEMORIA DE CADA IMAGEN
MOSTRAR UN EJEMPLO DE CADA TIPO

%Briefing of features for selected softwares
%Imperceptibility, capacity and robustness assessment

\section{Results}

%Show imperceptibility, capacity and robustness results

\section{Discussion and Conclusion}





\end{document}
